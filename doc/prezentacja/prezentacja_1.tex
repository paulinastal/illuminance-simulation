\documentclass[a4paper,12pt]{article}
\usepackage[polish]{babel}
\usepackage[utf8]{inputenc}
\usepackage[T1]{fontenc}
\usepackage[top=1.5cm, bottom=2.0cm, left=2.5cm, right=2.5cm]{geometry}
\usepackage{indentfirst}
\usepackage{enumerate}
\usepackage{amssymb}



\title{Symulacja natężenia światła}
\author{Paulina Stal, Patrycja Marchwica}
\date{15.05.2020}

\begin{document}
	\maketitle
	
	\section{Wprowadzenie}
	\label{sec:wprowadzenie}
	
	Celem symulacji będzie analiza natężenia oświetlenia w zamodelowanym pomieszczeniu -- sali lekcyjnej. Do przeprowadzenia symulacji zostanie wykorzystany pakiet \emph{VI-Suite}, czyli zintegrowany zestaw narzędzi do analizy otoczenia, wykorzystujący wbudowane funkcje programu do modelowania 3D jakim jest \emph{Blender} oraz integrujący zewnętrzne aplikacje tj. \emph{Radiance}, które umożliwiają przeprowadzenie symulacji oświetlenia.
	
	Pomiar natężenia światła, czyli gęstości strumienia świetlnego padającego na daną powierzchnię, którego jednostką w układzie SI jest luks [lx], zostanie wykonany w różnych miejscach zamodelowanego pomieszczenia. Otrzymane wyniki zostaną poddane analizie, mającej na celu określenie wpływu warunków pogodowych, konfiguracji opraw oświetleniowych oraz mocy świecenia opraw oświetleniowych na przebieg symulacji.  
	
	\section{Parametry modelu}
	\label{sec:parametry_modelu}
	
	\begin{itemize}
		\item Model sali lekcyjnej
		\begin{itemize}
			\item Wymiary modelu sali lekcyjnej -- $6m$ x $10m$ x $2.5m$
			\item Pole powierzchni podłogi -- $60m^2$
			\item Wymiary ławek - $500mm$ x $1300mm$ x $40mm$
			\item Wysokość ławek ($640mm$) i krzeseł ($38mm$) dostosowana do wzrostu ucznia -- $133-159cm$
		\end{itemize}
		\item Okna
		\begin{itemize}
			\item Stosunek powierzchni okien do powierzchni podłogi -- $1:5$
			\item Wymiary okien -- $1m$ x $1.5m$ x $0.12m$
			\item Odległość pomiędzy oknami -- $1.15m$
			\item Odległość okna od podłogi -- $0.68m$
			\item Okna znajdują się od wschodniej i zachodniej strony pomieszczenia
		\end{itemize}
		\item Oświetlenie
		\begin{itemize}
			\item Typ oświetlenia -- Oświetlenie LED natynkowe
			\item Wymiary opraw oświetleniowych -- $620mm$ x $620mm$ x $66mm$
			\item Temperatura barwowa -- $4000K$
			\item Rozmieszczenie opraw oświetleniowych 1:
			\begin{itemize}
				\item 8 lamp -- w dwóch rzędach po 4 wzdłuż sali lekcyjnej
				\item Odległość pomiędzy oprawami -- $1.13m$
				\item Odległość pomiędzy oprawą a oknami -- $1.48m$
				\item Odległość pomiędzy oprawą a ścianami północną i południową -- $1.43m$
			\end{itemize}
			\item Rozmieszczenie opraw oświetleniowych 2:
			\begin{itemize}
				\item
			\end{itemize}
		\end{itemize}
		\item Parametry Vi-Suite
		\begin{itemize}
			\item Lokalizacja -- Kraków -- dane zawarte w pliku EnergyPlus weather pobranym z [11]
			\item Pomiar natężenia oświetlenia co 1h pomiędzy 8.00 a 16.00
		\end{itemize}		
		
	\end{itemize}
	
	
	\section{Plan działania}
	\label{sec:plan_dzialania}
	\begin{enumerate}
		\item Wykonanie modelu sali lekcyjnej przy użyciu programu \emph{Blender} \checkmark
		\item Wykonanie modelu natynkowej oprawy oświetleniowej zgodnej ze specyfikacją, przy użyciu programu \emph{Blender} \checkmark
		\item Wstępny dobór konfiguracji opraw oświetleniowych \checkmark
		\item Nadanie parametrów/materiałów Vi-Suite obiektom biorącym udział w sumlacji \checkmark
		\item Sporządzenie schematu w Node Edytorze \checkmark
		\item Dobór parametrów przeprowadzanych symulacji \checkmark
		\item Przeprowadzenie symulacji przy użyciu pakietu \emph{VI-Suite} i zewnętrznej aplikacji \emph{Radiance} dla różnych konfiguracji wybranych parametrów:
		
		
		\begin{table}[h]
			\begin{center}
				\begin{tabular}{|c|c|c|c|c|c|c|c|c|c|c|c|c|c|c|}
					\cline{4-15}
					\multicolumn{3}{c|}{}&\multicolumn{6}{c|}{Z oświetleniem (1)} & \multicolumn{6}{c|}{Z oświetleniem (2)}\\ \cline{4-15}
					\multicolumn{3}{c|}{} & \multicolumn{12}{c|}{Radiancja $[W/sr/m^2]$} \\ \cline{2-15}
					\multicolumn{1}{c|}{}&\multicolumn{2}{c|}{Bez oświetlenia}& 10&15&20&10&15&20 & 10&15&20&10&15&20\\ \hline
					Słonecznie & \checkmark & \checkmark & \checkmark & \checkmark &\checkmark&x&x&x&&&&x&x&x\\ \hline
					Częściowe zachmurzenie  & \checkmark & \checkmark & \checkmark & \checkmark &\checkmark& \checkmark & \checkmark & \checkmark &&&&&&  \\ \hline
					Całkowite zachmurzenie  & \checkmark & \checkmark & \checkmark & \checkmark & \checkmark & \checkmark & \checkmark & \checkmark &&&&&& \\ \hline
					\multicolumn{1}{c|}{}& Luty & Maj & \multicolumn{3}{c|}{Luty} & \multicolumn{3}{c|}{Maj} &  \multicolumn{3}{c|}{Luty} & \multicolumn{3}{c|}{Maj}\\ \cline{2-15}
					
				\end{tabular}
			\end{center}
		\end{table}
		
		
		\item Zbiorcze zestawienie i analiza otrzymanych wyników 
		\item Porównanie rezultatów z rzeczywistymi badaniami
		\item Wnioski i podsumowanie
		
	\end{enumerate}
	
	\section{Przegląd literatury}
	\label{sec:przeglad_literatury}
	\begin{thebibliography}{7}
		\bibitem{1}
		D.Heim, A. Kujawski, \textit{Rozkład natężenia oświetlenia dziennego dla prostych struktur zabudowy}
		\bibitem{2}
		K. Błażejczyk et al., \textit{Seasonal and regional differences in lighting conditions and their influence on melatonin secretion}, Quaestiones Geographicae, 33(3), 2014, 17--25
		\bibitem{3}
		M. Ayoub, \textit{A review on light transport algorithms and simulation tools to model daylighting inside buildings}, Solar Energy, 198, 2020, 623--642
		\bibitem{4}
		L. Bellia, F. Fragliasso, \textit{Automated daylight-linked control systems performance with illuminance sensors for side-lit offices in the Mediterranean area}, Automation in Construction, 100 , 2019, 145--162
		\bibitem{5}
		R. Southall, F. Biljecki, \textit{The VI-Suite: a set of environmental analysis tools with geospatial data applications}, Open Geospatial Data, Software and Standards, 2017, 2--23
		\bibitem{6}
		\textit{Recommended Light Levels (Illuminance) for Outdoor and Indoor Venues}
		\bibitem{7}
		V. Logar, Z. Kristl, I. Skrjanc, \textit{Using a fuzzy black-box model to estimate the indoor illuminance in buildings}, Energy and Buildings, 70, 2014, 343--351
		\bibitem{8}
		Jerzy Wójcik Architekt IARP, "Szkoły podstawowe,szkoły średnie. Technologia obiektów", 28.10.2014
		\bibitem{9}
		http://www.instalacjebudowlane.pl/9490-29-12421-oswietlenie-led-w-szkolach--normy-i-rozwiazania-praktyczne.html
		\bibitem{10}
		Dr Ryan Southall, School of Art, Design \& Media - University of Brighton, Simulations and Visualisations with the VI-Suite - For VI-Suite Version 0.4
		\bibitem{11}
		https://energyplus.net/weather
		
	\end{thebibliography}
	
\end{document}