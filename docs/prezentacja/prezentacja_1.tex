\documentclass[a4paper,12pt]{article}
\usepackage[polish]{babel}
\usepackage[utf8]{inputenc}
\usepackage[T1]{fontenc}
\usepackage[top=1.5cm, bottom=2.0cm, left=2.5cm, right=2.5cm]{geometry}
\usepackage{indentfirst}
\usepackage{enumerate}



\title{Symulacja natężenia światła}
\author{Paulina Stal, Patrycja Marchwica}
\date{8.04.2020}

\begin{document}
\maketitle

	\section{Wprowadzenie}
	\label{sec:wprowadzenie}
	
	Celem symulacji będzie analiza natężenia oświetlenia w zamodelowanym pomieszczeniu. Do przeprowadzenia symulacji zostanie wykorzystany pakiet \emph{VI-Suite}, czyli zintegrowany zestaw narzędzi do analizy otoczenia, wykorzystujący wbudowane funkcje programu do modelowania 3D jakim jest \emph{Blender} oraz integrujący zewnętrzne aplikacje tj. \emph{Radiance}, które umożliwiają przeprowadzenie symulacji oświetlenia.

	Pomiar natężenia światła, czyli gęstości strumienia świetlnego padającego na daną powierzchnię, którego jednostką w układzie SI jest luks [lx], zostanie wykonany w różnych miejscach zamodelowanego pomieszczenia. Otrzymane wyniki zostaną poddane analizie, mającej na celu określenie wpływu warunków pogodowych, konfiguracji opraw oświetleniowych oraz mocy świecenia opraw oświetleniowych na przebieg symulacji.  

	
	\section{Przegląd literatury}
	\label{sec:przeglad_literatury}
		\begin{thebibliography}{7}
			\bibitem{1}
			D.Heim, A. Kujawski, \textit{Rozkład natężenia oświetlenia dziennego dla prostych struktur zabudowy}
			\bibitem{2}
			K. Błażejczyk et al., \textit{Seasonal and regional differences in lighting conditions and their influence on melatonin secretion}, Quaestiones Geographicae, 33(3), 2014, 17--25
			\bibitem{3}
			M. Ayoub, \textit{A review on light transport algorithms and simulation tools to model daylighting inside buildings}, Solar Energy, 198, 2020, 623--642
			\bibitem{4}
			L. Bellia, F. Fragliasso, \textit{Automated daylight-linked control systems performance with illuminance sensors for side-lit offices in the Mediterranean area}, Automation in Construction, 100 , 2019, 145--162
			\bibitem{5}
			R. Southall, F. Biljecki, \textit{The VI-Suite: a set of environmental analysis tools with geospatial data applications}, Open Geospatial Data, Software and Standards, 2017, 2--23
			\bibitem{6}
			\textit{Recommended Light Levels (Illuminance) for Outdoor and Indoor Venues}
			\bibitem{7}
			V. Logar, Z. Kristl, I. Skrjanc, \textit{Using a fuzzy black-box model to estimate the indoor illuminance in buildings}, Energy and Buildings, 70, 2014, 343--351
			
		\end{thebibliography}
	
	\section{Plan działania}
	\label{sec:plan_dzialania}
		\begin{enumerate}
			\item Wykonanie modelu pomieszczenia 3D korzystając z programu \emph{Blender}.
			\item Opracowanie modelu oprawy oświetleniowej.
			\item Zastosowanie pakietu \emph{VI-Suite} do pomiaru oświetlenia, zwracając uwagę na:
			\begin{itemize}
				\item warunki pogodowe,
				\item porę dnia,
				\item konfigurację opraw oświetleniowych,
				\item moc świecenia.
   			\end{itemize}
			\item Przedstawienie wyników pomiarów.
			\item Porównanie rezultatów z rzeczywistymi badaniami.
			\item Zapisanie wniosków.
			\item Podsumowanie.
		\end{enumerate}
	
	\section{Pytania i wątpliwości}
	\label{sec:pytania_watpliwosci}

	
\end{document}